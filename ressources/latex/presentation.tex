\documentclass{beamer}

\usepackage[frenchb]{babel}
\usepackage[T1]{fontenc}
\usepackage[utf8]{inputenc}
\usepackage{hyperref}

\usetheme{Singapore}

\title{Codes correcteurs et cryptosystème de Mc Eliece}
\author{Auclair Pierre}

\newtheorem{prop}{Propriété}
\newtheorem{algo}{Algorithme}
\newtheorem{defi}{Définition}
\newtheorem{pb}{Problème}

\begin{document}

	\begin{frame}{Introduction}
		\tableofcontents
		
	\end{frame}

	\section{Codes de Goppa}

		\subsection{Notations}

			\begin{frame}{Notations}

				\begin{itemize}
					\item $f \in L(\mathbb{F}^{k},\mathbb{F}^{n})$ avec k < n \\
					\item $ G=Mat(f) $ matrice génératrice $n\times k$\\
					\item $Ker(H) = Im(f)$ matrice de parité $k\times n$\\
					\item $ S_{y} : y \in \mathbb{F}^{n} \rightarrow Hy \in \mathbb{F}^{k} $ syndrome \\
					\item $ \omega : y \in \mathbb{F}^{n} \rightarrow Card( i / y_{i}\neq 0 ) $ poids de Hamming
					\item d la distance minimale du code
				\end{itemize}

			\end{frame}

		\subsection{Définition}

			\begin{frame}{Définition}

				\begin{itemize}
					\item g polynôme de $ \mathbb{F}_{2^m}[X]$ irréductible
					\item L = $ (\alpha_{1},..,\alpha_{n})  \in \mathbb{F}_{2^{m}}^{n}$ le support
				\end{itemize}

				On définit un code de Goppa par son syndrome
				$$\mathbf{S}_{y}(x) = \sum_{i=1}^{n} \frac{y_{i}}{x-\alpha_{i}} \ mod \ g(x)$$

				En pratique on prendra $n = 2^{m}$

			\end{frame}

		\subsection{Matrice de parité}

			\begin{frame}{Matrice de parité}

				$$
					\frac{1}{{x-\alpha_{i}}} = \frac{1}{g(\alpha_{i})} \sum_{k=0}^{t-1} x^{k}\sum_{j=k+1}^{t} g_{j}\alpha_{i}^{j-1-k}
				$$
				De là on déduit une expression d'une matrice de parité.
				$$
				\begin{pmatrix}
					\frac{1}{g(\alpha_{1})} &  \frac{1}{g(\alpha_{2})} & \cdots &  \frac{1}{g(\alpha_{n})} \\
					\frac{1}{g(\alpha_{1})}g_{t}\alpha_{1} &  \frac{1}{g(\alpha_{2})}g_{t}\alpha_{2} & \cdots &  \frac{1}{g(\alpha_{n})}g_{t}\alpha_{n}\\
					\vdots & \vdots & \vdots & \vdots\\
					\frac{1}{g(\alpha_{1})}g_{t}\alpha_{1}^{t-1} &  \frac{1}{g(\alpha_{2})}g_{t}\alpha_{2}^{t-1} & \cdots &  \frac{1}{g(\alpha_{n})}g_{t}\alpha_{n}^{t-1} \\
				\end{pmatrix}
				$$
						
			\end{frame}

		\subsection{Décodage}

			\begin{frame}{Décodage}
				Soit le polynôme localisateur d'erreurs $\sigma$
				$$
					\sigma_{y}(x) = \prod \epsilon_{i}(x-\alpha_{i})
				$$
				Il faut résoudre l'équation clef suivante :
				$$
					\omega_{y}(x) = S_{y}(x) \sigma_{y}(x) \ mod \ g
				$$
				On montre son unicité.
				On explicite une solution avec l'algorithme d'Euclide étendu.
			\end{frame}

	\section{Mc Eliece}

		\subsection{Clefs}

			\begin{frame}{Clefs}
				Clef privée
				\begin{itemize}
					\item G matrice génératrice, ainsi que L et g
					\item P matrice de permutation $ n \times n$
					\item Q $\in GL_{k}(\mathbb{F})$
				\end{itemize}

				Clef publique
				\begin{itemize}
					\item G' = PGQ
					\item capacité de correction
				\end{itemize}
			
			\end{frame}

		\subsection{Principe}

			\begin{frame}{Principe}
				
				\begin{itemize}
					\item $x \in \mathbb{F}^{k}$ le message a envoyer
					\item $G'x + \epsilon = PGQx + \epsilon$ message envoyé
					\item $ GQ + P^{-1}\epsilon$ peut être corrigé
					\item Connaissant Q on déduit x le message initial
				\end{itemize}

				Sécurité par 2 problèmes de théorie des codes :
				\begin{itemize}
					\item Indistinguabilité d'un code de Goppa
					\item Décodage par syndrome NP-complet
				\end{itemize}

			\end{frame}

	\section{Mise en œuvre}

		\subsection{POO}

			\begin{frame}{Programmation Orientée objet}
				Rédaction de classes :
				\begin{itemize}
					\item Corps de galois $\mathbb{F}_{p^{m}}$
					\item Optimisation en binaire pour p=2
					\item Matrices
					\item Polynomes
					\item Clefs publiques et clefs privées
				\end{itemize}
			\end{frame}

		\subsection{Berlekamp-Hensel}

			\begin{frame}{Algorithme de Berlekamp-Hensel}

				Teste l'irréductibilité de $ g \in \mathbb{F}_{2^{m}}[X] $
				$$
					\exists P / 1<deg(P)<deg(g) \ et \ P^{2^{m}} - P = 0 \ mod \ g \Leftrightarrow g \ reductible
				$$
				Par le morphisme de Frobenius, on regarde le noyau de l'application linéaire :
				$$
					P \leftarrow P^{2^{m}} - P
				$$
				
			\end{frame}
		
	    

\end{document}