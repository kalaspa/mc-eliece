\documentclass{beamer}

\usepackage[frenchb]{babel}
\usepackage[T1]{fontenc}
\usepackage[utf8]{inputenc}
\usepackage{hyperref}

\usetheme{Singapore}

\title{Codes correcteurs et cryptosystème de Mc Eliece}
\author{Auclair Pierre}

\newtheorem{prop}{Propriété}
\newtheorem{algo}{Algorithme}
\newtheorem{defi}{Définition}
\newtheorem{pb}{Problème}

\begin{document}

	\begin{frame}{Introduction}
		\tableofcontents
		
	\end{frame}

	\section{Codes de Goppa}

		\subsection{Notations}

			\begin{frame}{Notations}

				\begin{tabular}{ll}
					$f \in L(\mathbb{F}^{k},\mathbb{F}^{n})$ & avec $k < n$ \\
					$ G=Mat(f) $ & $G$ matrice génératrice $n\times k$\\
					$Ker(H) = Im(f)$ & $H$ matrice de parité $k\times n$\\
					$ S_{y}(X) = 0 \Leftrightarrow y \in Im(f) $ & $S_{y}(X)$ syndrome \\
					$ d : y \in \mathbb{F}^{n} \rightarrow Card( i / y_{i}\neq 0 ) $ & $d$ poids de Hamming
				\end{tabular}

			\end{frame}

		\subsection{Définition}

			\begin{frame}{Définition}

				\begin{itemize}
					\item $g$ polynôme de $ \mathbb{F}_{2^m}[X]$ irréductible de degré $t$.
					\item $L$ = $ (\alpha_{1},..,\alpha_{n})  \in \mathbb{F}_{2^{m}}^{n}$ le support
				\end{itemize}

				On définit un code de Goppa par son syndrome
				$$\mathbf{S}_{y}(x) = \sum_{i=1}^{n} \frac{y_{i}}{x-\alpha_{i}} \ mod \ g(x)$$
				\begin{itemize}
					\item En pratique on prendra $n = 2^{m}$.
					\item Capacité de correction $ \geq \frac{t}{2}$
				\end{itemize}

			\end{frame}

		\subsection{Décodage}

			\begin{frame}{Décodage}
				\begin{itemize}
					\item $\sigma_{y}(X)$ polynôme localisateur d'erreurs
					\item $z = y + \epsilon \in \mathbb{F}^{n} $ avec $y$ mot de code
					\item $L = (\alpha_{1}, .. ,\alpha{n})$ le support
				\end{itemize}
				$$
					\sigma_{z}(X) = \prod (X-\alpha_{i})^{\epsilon_{i}}
				$$
				Soit l'équation clef suivante :
				$$
					\omega_{z}(X) = S_{z}(X) \sigma_{z}(X) \ mod \ g
				$$
				\begin{itemize}
					\item On montre que $deg(\omega_{z}(X) < \frac{t}{2}) $
					\item Unicité du couple solution avec les contraintes de degrés
					\item Existence grâce à l'algorithme d'Euclide étendu
				\end{itemize}

			\end{frame}

	\section{Mc Eliece}

		
		\subsection{Clefs}

			\begin{frame}{Clefs}
				Clef privée
				\begin{itemize}
					\item G matrice génératrice, ainsi que L et g
					\item P matrice de permutation $ n \times n$
					\item Q $\in GL_{k}(\mathbb{F})$
				\end{itemize}

				Clef publique
				\begin{itemize}
					\item G' = PGQ
					\item capacité de correction
				\end{itemize}
			
			\end{frame}

		\subsection{Principe}

			\begin{frame}{Principe}

				Soit $x \in \mathbb{F}^{k}$ le message à envoyer de Alice vers Bob

				\begin{itemize}
					\item Bob partage $G'=PGQ$ et une capacité de correction
					\item Alice envoie $y = PGQx + \epsilon$
					\item Bob calcule $P^{-1}y = GQx + P^{-1}\epsilon $
					\item Bob corrige l'erreur et obtient $Qx$
					\item Bob obtient $x$ en connaissant Q inversible
				\end{itemize}

				Sécurité par 2 problèmes de théorie des codes :
				\begin{itemize}
					\item Indistinguabilité d'un code de Goppa
					\item Décodage par syndrome NP-complet
				\end{itemize}

			\end{frame}

		

	\section{Mise en œuvre}

		\subsection{POO}

			\begin{frame}{Programmation Orientée objet}
				Rédaction de classes :
				\begin{itemize}
					\item Corps de Galois $\mathbb{F}_{p^{m}}$
					\item Optimisation en binaire pour $p=2$
					\item Matrices
					\item Polynômes
				\end{itemize}

				Intérêts : Pédagogique, flexibilité de Python.

				Inconvénients : Trop lent, peu optimisé par rapport aux bibliothèques standards.

			\end{frame}

		\subsection{Berlekamp-Hensel}

			\begin{frame}{Algorithme de Berlekamp-Hensel}

				Teste l'irréductibilité de $ g \in \mathbb{F}_{2^{m}}[X] $
				$$
					\exists P / 1<deg(P)<deg(g) \ et \ P^{2^{m}} - P = 0 \ mod \ g \Leftrightarrow g \ reductible
				$$
				Par le morphisme de Frobenius, on regarde le noyau de l'application linéaire :
				$$
					P \leftarrow P^{2^{m}} - P
				$$
				
			\end{frame}



		\begin{frame}{Bilan}
			
			Programme fonctionnel mais lent :
			\begin{itemize}
				\item Paramètres conseillés $ (n,k,t) = (1024,524,50) $
				\item 300s pour générer les clefs de paramètres $ (256 ,184 ,9 ) $
			\end{itemize}

			Possibilités d'optimisation :
			\begin{itemize}
				\item Usage de Sage pour le calcul matriciel : 100s pour $(1024,524,20)$
				
			\end{itemize}


		\end{frame}
		
	    

\end{document}