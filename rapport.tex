\documentclass{article}

\usepackage[utf8]{inputenc}
\usepackage{lmodern}
\usepackage[francais]{babel}
\usepackage{amsfonts} %permet d'utiliser mathbb
\usepackage{amsthm} %permet d'utiliser \newtheorem*
 \usepackage{amsmath} %permet d'utiliser les matrices
\usepackage[colorlinks=true,linkcolor=black]{hyperref}
\usepackage{listings}
\usepackage{url}


\usepackage[top=2cm, bottom=2cm, left=3cm, right=3cm]{geometry}

\lstset {language = Python,
	numbers = left,
	showspaces = false,
	tabsize = 2,
	showstringspaces = false
	}

\title{Codes correcteurs et cryptosystème de Mc Eliece}
\author{Auclair Pierre}

\begin{document}
	\maketitle

	Le cryptostème de Mc Eliece est un système de cryptage qui s'appuie sur la théorie des codes correcteurs.
	Il se situe à l'intersection des soucis de fiabilité et de sécurité de l'information.
	C'est pourquoi nous avons implémenté ce système en python dans le cadre du sujet transfert et échange.


	\part*{Bagage}

		\section{Théorie des codes}
			%def, principe
			%Principe

		\section{Outils informatiques}
			% bibliothèques matrices, corps finis, polynomes

		\section{Algorithmes supplémentaires}
			%Gauss
			%Berlekamp-Hensel

	\part*{Codes de Goppa}

		\section{Définition}
			%Syndrome

		\section{Décodage}
			%Equation clef

		\section{Implémentation}
			

	\part*{Mc Eliece}

		\section{Principe}
			%Clefs, codage, décodage

		\section{Sécurité}
			 %Syndrome decoding et indistingabilité

		\section{Mise en oeuvre}
			%Classe clef






\bibliographystyle{alpha}
\bibliography{ref}
\end{document}
